\section{Conclusion}

\subsection*{Summary of Key Findings}

Our experiments indicate that:

\begin{itemize}
    \item The \textbf{BSI Algorithm} offers a good balance between performance and computational efficiency, performing close to optimal across various instance types.
    \item The \textbf{Greedy Algorithm}, while fast, may not provide acceptable solutions for instances with non-uniform weight distributions.
    \item The \textbf{$k$-Lookahead Algorithm} improves solution quality as $k$ increases but at the cost of increased computational time, making it less practical for large instances.
\end{itemize}

\subsection*{Practical Implications}

For practical applications:

\begin{itemize}
    \item \textbf{Use BSI} when dealing with instances where job weights vary significantly.
    \item \textbf{Greedy Algorithm} may suffice for instances with uniform weights or when computational resources are limited.
    \item \textbf{$k$-Lookahead} is suitable when solution quality is critical, and computational time is less of a concern.
\end{itemize}

\subsection*{Reflection}

The study underscores the importance of selecting appropriate algorithms based on instance characteristics. Understanding the weight distribution and how it interacts with local precedence constraints is crucial for achieving efficient scheduling.