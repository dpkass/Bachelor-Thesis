\begin{abstract}
    Scheduling plays a critical role in optimizing resource utilization across various industries, including manufacturing, logistics, and computing. This thesis investigates a specific variant of the scheduling problem characterized by \textbf{local precedence constraints}, where precedence requirements are enforced only within individual machines based on a global job order—a scenario that has received limited attention in existing literature. The primary objective is to \textbf{minimize the weighted sum of completion times}, thereby prioritizing higher-weighted jobs to enhance efficiency and cost-effectiveness.

    To address this problem, we develop several algorithms, including an exact \textbf{dynamic programming} method and various heuristic and approximation techniques designed for practical application in large-scale instances. A comprehensive experimental framework is established to evaluate these algorithms across diverse scenarios, varying in the number of machines and job weight distributions.

    The findings highlight the conditions under which specific \textbf{algorithms} perform optimally or near-optimally, providing insights into their strengths and limitations. For instance, certain algorithms demonstrate exceptional efficiency in handling instances with highly skewed weight distributions, while others excel in scenarios with ordered weights. Additionally, the research explores the trade-offs between computational complexity and solution quality, noting that some algorithms achieve faster runtimes at the expense of optimality, whereas others deliver highly optimal solutions with increased computational demands.

    These contributions advance the understanding of scheduling with local precedence constraints, addressing a notable gap in existing research. The results offer a robust foundation for future studies and practical implementations, paving the way for more efficient and effective scheduling strategies in relevant industries.
\end{abstract}
