\chapter{Introduction}

Scheduling is a cornerstone problem in operations research and computer science, pivotal for optimizing the utilization of resources across various industries such as manufacturing, logistics, and computing. Efficient scheduling algorithms are essential for minimizing operational costs, reducing processing times, and enhancing overall productivity. In manufacturing, for instance, effective scheduling ensures that production lines operate smoothly, minimizing downtime and maximizing output. Similarly, in logistics, optimal scheduling of deliveries can lead to significant cost savings and improved service quality. In the realm of computing, especially in parallel and distributed systems, scheduling tasks efficiently is crucial for maximizing computational throughput and minimizing latency.

This thesis focuses on a specific variant of the scheduling problem characterized by \textbf{local precedence constraints}. In this context, precedence constraints must be respected only within each machine, rather than globally across all machines. This means that while each machine has its own set of job orderings that must adhere to a predefined global sequence, there are no dependencies between jobs assigned to different machines. Such constraints are prevalent in real-world scenarios. For example, in manufacturing processes, certain tasks on the same machine must follow a specific order to ensure product quality and operational efficiency. Similarly, in parallel computing, tasks assigned to the same processor may have dependencies that necessitate a particular execution sequence to maintain system stability and performance.

The primary objective in this scheduling problem is to \textbf{minimize the weighted sum of completion times}. This metric is a reflection of both efficiency and cost-effectiveness, as it accounts for the importance or cost associated with each job. By minimizing this sum, the schedule ensures that higher-weighted (more critical or costly) jobs are completed earlier, thereby optimizing overall performance and resource utilization. \hfill \break

However, the introduction of local precedence constraints adds a layer of complexity to the scheduling problem. Balancing the load across multiple machines while respecting these orderings of tasks requires sophisticated algorithmic strategies. The challenge lies in navigating the combinatorial explosion of possible job assignments and sequences, especially as the number of jobs and machines increases.

The landscape of scheduling problems is vast, encompassing various models and constraints that cater to different real-world applications. Classical scheduling problems, such as \textbf{single-machine scheduling}, have been extensively studied, with numerous algorithms developed to find optimal or near-optimal solutions \cite{ChekuriMotwani:99:Precedence-constrained}. These foundational studies provide a basis for understanding more complex scheduling scenarios.

In \textbf{multi-machine scheduling}, the complexity of effectively distributing jobs across machines becomes substantially greater. Greedy algorithms and list scheduling are commonly used heuristics that strive to balance the load while minimizing performance metrics like makespan. Despite their simplicity and efficiency in many scenarios, these heuristics often fall short when precedence constraints are introduced. Global precedence constraints, which mandate that certain jobs must be completed before others across all machines, introduce intricate dependencies that complicate the scheduling process. Its NP-hardness has long been established, and even the best polynomial-time approximation algorithms for this problem face limitations, with approximation ratios no better than $\frac{4}{3}$ \cite{LenstraRinnooy-Kan:78:Complexity-of-scheduling}. This underscores the significant computational challenges posed by precedence constraints in multi-machine scheduling.

Existing literature on scheduling with precedence constraints encompasses both exact algorithms and approximation methods. Exact algorithms, including dynamic programming and branch-and-bound techniques, strive to find optimal solutions but often suffer from high computational costs, making them impractical for large instances. On the other hand, approximation algorithms offer solutions that are provably close to optimal within certain bounds, providing a balance between solution quality and computational efficiency. Heuristic methods, such as genetic algorithms and simulated annealing, have also been explored for their ability to find good-enough solutions in a reasonable timeframe, especially when dealing with complex or large-scale scheduling problems.

Despite the extensive research, there remains a notable gap in the study of scheduling problems with \textbf{local precedence constraints}. Most existing studies focus on global precedence constraints or scenarios without any precedence relations. The specific case where precedence constraints are localized within each machine, based on a global job order, has not been investigated. This gap highlights the need for specialized algorithms and analytical approaches to address the unique challenges posed by local precedence constraints. Additionally, the computational complexity of this problem variant remains \textbf{undetermined}, further emphasizing the necessity for in-depth research and exploration in this area.

The online platform \href{http://schedulingzoo.lip6.fr/}{The Scheduling Zoo}~\cite{SchedulingZoo} provides interactive tools for experimenting with various scheduling problem configurations and offer access to related research papers. They also provide a comprehensive overview of complexity analyses and the best known polynomial-time approximation algorithms (or the absence thereof). However, \textit{The Scheduling Zoo} currently does not include configurations for scheduling with local precedence constraints, thereby underscoring the existing gap in research and the need for specialized algorithms to address this particular variant. \hfill \break

This thesis presents significant advancements in the study of scheduling with local precedence constraints, addressing gaps in existing research and contributing to a deeper understanding of this complex problem. Central to this work is the development and evaluation of a diverse suite of algorithms specifically designed for scheduling under local precedence constraints. These include exact methods, such as \textbf{dynamic programming}, and \textbf{heuristic and approximation techniques} aimed at providing practical solutions for large-scale instances.

To support this algorithmic innovation, we designed a comprehensive \textbf{experimental framework} that evaluates algorithm performance across a wide variety of instance types. These instances vary in terms of machine numbers and weight distributions, ensuring a rigorous and thorough assessment of the proposed methods under diverse conditions. This experimental setup enables the systematic analysis of algorithmic efficacy, scalability, and stability, providing valuable insights into their strengths and limitations.

The findings from our extensive experimentation highlight key aspects of algorithm performance. We identified the conditions under which specific algorithms perform optimally or near-optimally, such as scenarios with varying machine numbers or distinct weight distributions. This enables practitioners to select the most suitable algorithm based on the characteristics of the problem instance. Furthermore, the study reveals notable performance trends. For instance, certain algorithms demonstrate exceptional efficiency in handling instances with highly skewed weight distributions, while others are better suited for scenarios with ordered weights. Understanding these trends is crucial for informed decision-making in applying scheduling algorithms to real-world problems.

Additionally, the research provides critical insights into the trade-offs between computational complexity and solution quality. While some algorithms prioritize faster runtimes at the expense of optimality, others achieve highly optimal solutions with increased computational demands. These trade-offs underscore the importance of balancing efficiency and effectiveness when selecting algorithms for practical applications. Together, these contributions advance the field of scheduling with local precedence constraints, offering a robust foundation for future research and development. \hfill \break

The thesis is systematically organized to present these findings in a coherent and logical manner. Following this introductory chapter, Chapter 2 delves into the formal problem definition and the dynamic programming approach developed to address it. Chapter 3 outlines the experimental setup and methodology used for evaluating the algorithms, and presents the experimental results, accompanied by a detailed analysis. Finally, Chapter 4 concludes the thesis, summarizing the key contributions and suggesting avenues for future research.
