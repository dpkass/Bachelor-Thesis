\chapter{Introduction}

\begin{markdown}
## Motivation
- Introduce scheduling as a fundamental problem in operations research and computer science.
  - Emphasize its importance in optimizing resources in various industries like manufacturing,
    logistics, and computing.
- Define the specific scheduling problem with **local precedence constraints**.
  - Explain that in this context, precedence constraints must be respected only within each
    machine.
  - Provide real-world examples:
    - Manufacturing processes where tasks on the same machine must follow a specific order.
    - Parallel computing where tasks assigned to the same processor have dependencies.
- Highlight the objective of minimizing the **weighted sum of completion times**.
  - Discuss how this metric reflects efficiency and cost-effectiveness.
- Explain the challenges posed by local precedence constraints.
  - Complexity added due to partial ordering of tasks.
  - Difficulty in balancing load across multiple machines while respecting local constraints.

## Related Work
- Review classical scheduling problems and algorithms.
  - Single-machine scheduling without precedence constraints.
  - Multi-machine scheduling and common heuristics.
  - Global precedence constraints and their impact on scheduling complexity.
- Discuss existing literature on scheduling with precedence constraints.
  - Exact algorithms (e.g., dynamic programming, branch and bound).
  - Approximation algorithms and known performance ratios.
  - Heuristic methods like greedy algorithms and their typical use cases.
- Identify gaps in the current research.
  - Limited focus on local (machine-specific) precedence constraints.
  - Lack of efficient algorithms for large-scale instances with local constraints.
  - Insufficient empirical analysis on different instance types and weight distributions.

## Our Results
- **Main Contributions:**
  - Developed and implemented several algorithms tailored for scheduling with local precedence
    constraints.
  - Designed a comprehensive experimental setup to evaluate algorithm performance across
    various
    instance types and machine counts.
  - Analyzed the performance, scalability, and stability of each algorithm.
- **Key Findings:**
  - Identified which algorithms perform optimally or near-optimally under specific conditions.
  - Discovered trends in algorithm performance relative to instance characteristics (e.g.,
    weight
    distributions).
  - Provided insights into the trade-offs between computational complexity and solution
    quality.
- **Thesis Structure Overview:**
  - Outline how the thesis is organized to present these findings systematically.
\end{markdown}
